\documentclass[10pt,
               a4paper,
               journal,
               ]{IEEEtran}
\makeatletter

\def\markboth#1#2{\def\leftmark{\@IEEEcompsoconly{\sffamily}\MakeUppercase{\protect#1}}%
\def\rightmark{\@IEEEcompsoconly{\sffamily}\MakeUppercase{\protect#2}}}
\makeatother

\usepackage[utf8]{inputenc}
\usepackage[T1]{fontenc}
\usepackage{cite}
\usepackage[pdftex]{graphicx}
\graphicspath{{../png/}}
\DeclareGraphicsExtensions{.png}
\usepackage[cmex10]{amsmath}
\interdisplaylinepenalty=2500
\usepackage{algorithmic}
\usepackage{array}
\usepackage{mdwmath}
\usepackage{mdwtab}
\usepackage{eqparbox}
\usepackage[caption=false,font=footnotesize]{subfig}
\usepackage{fixltx2e}
\usepackage{stfloats}
\usepackage{url}
\hyphenation{op-tical net-works semi-conduc-tor}

\begin{document}
	\title{Constraint Programming - Inside Gecode}
	\author{Benedikt~Schmidt}
	\markboth{Advanced Seminar for Security in Information Technology, Summer Term 2014}%
	{Benedikt Schmidt: Constraint Programming - Inside Gecode}	
	\maketitle	
	
	\begin{abstract}	
		Put the abstract here. The abstract should be 200-300 words long and must
		answer the following questions:
		\begin{itemize}
		   \item Why is this topic important and interesting?
		   \item What is the topic?
		\end{itemize}
		It's a good writing style to answer the "why" question first by putting the
		topic in a broader context. 	
	\end{abstract}
	
	\section{Introduction}
	In general a scientific paper starts with an introduction to the topic.
	
	\section{State of the Art}
	In this section an analysis of the available literature on the topic is done.
	This section may be split or subdivided into several sections or subsections.
	
	\subsection{Subsection Heading Here}
	Subsection text here.
	
	\subsubsection{Subsubsection Heading Here}
	Subsubsection text here.
	
	\section{Please Note}
	Regular research papers need at least two additional sections here. One section
	for contributions and methods and one section for the results. For seminar
	papers these sections can be omitted. 
	
	\section{Conclusion}
	Put the conclusions of the work here. The conclusion is like the abstract with
	an additional discussion of open points.
	
	\begin{thebibliography}{1}
		\bibitem{IEEEhowto:kopka}
		H.~Kopka and P.~W. Daly, \emph{A Guide to \LaTeX}, 3rd~ed.\hskip 1em plus
		  0.5em minus 0.4em\relax Harlow, England: Addison-Wesley, 1999.
	\end{thebibliography}
\end{document}


